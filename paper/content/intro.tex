\section{Introduction}

The \gls{ipa} uses special symbols\footnote{See for example the French list \href{https://en.wikipedia.org/wiki/Help:IPA/French}{here}.} to represent the sound of a spoken language \cite{ipa}. This is useful for language learners since the pronunciation of a word can be significantly different from its written form. For example, the French word \textit{renseignement} (information) is pronounced \textipa{/K\~{a}.sE\textltailn.m\~{a}/}. Based on this alphabet, one might wonder if we can construct a metric that quantifies the \textbf{similarity between two words based on their phonetic transcription}. This would allow to construct a graph where nodes are words and undirected edges are weighted by the distance between the words. Such a graph can be used to find neighbors of a word based on the respective phonetic similarity. This opens up the possibility to apply clustering algorithms and other methods stemming from graph theory in order to analyze the phonetic structure of a language.

Calculating the distance between each pair of words corresponds to a fully-connected graph. Our dataset consists of around 600,000 French words and their \gls{ipa} transcription, alongside their frequency in the French language. Excluding self-loops, we find a vast number of edges \eqref{eq:num-edges}:
\begin{align}
    \text{\#nodes} &= 600,000 \\
    \text{\#edges} &= \frac{600,000 \cdot 599,999}{2} \approx \num{1.8e11}
\end{align}
This high number and the independent nature of the distance calculation for each pair of words makes the problem well-suited for parallelization. In \autoref{sec:needleman-wunsch}, we present the Needleman-Wunsch algorithm used to calculate the distance between two words. In \autoref{sec:impl}, we discuss how to parallelize this algorithm on a CPU using the \textit{Rayon} library in Rust and on a consumer Nvidia GPU using the CUDA framework with the \textit{cudarc} Rust library. Finally, we present performance results and provide visualizations of the obtained graphs in \autoref{sec:eval}. We conclude in \autoref{sec:conclusion} and discuss how this method can be improved and how future work could extend it.

% In the following, by \textit{word} we always refer to its phonetic transcription. That is, homophones like the French words \textit{vert} \textipa{/vEK/} (green) and \textit{verre} \textipa{/vEK/} (glass) are considered the same word.

% \vfill\null
