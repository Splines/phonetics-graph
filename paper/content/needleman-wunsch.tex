\section{Needleman-Wunsch algorithm}
\label{sec:needleman-wunsch}

% \end{multicols*}{2}
\begin{algorithm*}
    \DontPrintSemicolon
    
    \SetKwFunction{calcScoreFunc}{calculateScore}
    \SetKwData{WordA}{$A$}
    \SetKwData{WordB}{$B$}
    \SetKwData{Sim}{similarity}
    \SetKwData{GapPenalty}{$p$}
    \SetKwData{ScoreMatrix}{scoreMatrix}
    \newcommand{\ScoreMatrixIdx}[2]{{\ScoreMatrix}[{#1}][{#2}]}
    \SetKwData{Cost}{cost}
    \SetKwData{MatchScore}{matchScore}
    \SetKwData{DeleteScore}{deleteScore}
    \SetKwData{InsertScore}{insertScore}
    \SetKwData{Score}{score}
    \SetKwFunction{len}{len}

    \KwIn{
        $\WordA = \set{A_0, \ldots, A_{\len(A)-1}}$,
        $\WordB = \set{B_0, \ldots, B_{\len(B)-1}}$,\\
        \qquad\qquad \Sim: similarityScoreFunc,
        \GapPenalty: GapPenalty
    }
    \KwOut{\Score}
    \Fn{\calcScoreFunc{}}
    {
        Init \ScoreMatrix with dimensions $(\len(\WordA)+1) \times (\len(\WordB)+1)$\;

        \BlankLine

        \For{$i \in \{0, \ldots, \len(\WordA)\}$}
        {
            \ScoreMatrixIdx{$i$}{$0$} $\gets \GapPenalty \cdot i$\;
        }
        \For{$j \in \{0, \ldots, \len(\WordB)\}$}
        {
            \ScoreMatrixIdx{$0$}{$j$} $\gets \GapPenalty \cdot j$\;
        }

        \BlankLine

        \For{$i \in \{1, \ldots, \len(\WordA)\}$}
        {
            \For{$j \in \{1, \ldots, \len(\WordB)\}$}
            {
                \Cost $\gets$ $\Sim(\WordA_{i}, \WordB_{j})$\;
                \MatchScore $\gets$ \ScoreMatrixIdx{$i-1$}{$j-1$} + \Cost\;
                \DeleteScore $\gets$ \ScoreMatrixIdx{$i-1$}{$j$} + \GapPenalty\;
                \InsertScore $\gets$ \ScoreMatrixIdx{$i$}{$j-1$} + \GapPenalty\;
                \ScoreMatrixIdx{$i$}{$j$} $\gets$ $\max(\MatchScore, \DeleteScore, \InsertScore)$\;
            }
        }

        \BlankLine

        \Return\! \ScoreMatrixIdx{$\len(\WordA)$}{$\len(\WordB)$}\;
    }
    
    \caption{Needleman-Wunsch}
    \label{alg:needleman-wunsch}
\end{algorithm*}
% \begin{multicols*}{2}

TODO... which calculates the global alignment of two sequences and was originally used in bio-informatics to compare DNA sequences. Here, the alphabet will instead consist of the phonetic symbols. Out of all possible alignments of two words (including gaps), the Needleman-Wunsch algorithm finds the one with the smallest distance, \ie the alignment with the highest score.

A good introduction to the algorithm can be found on the respective \href{https://en.wikipedia.org/wiki/Needleman%E2%80%93Wunsch_algorithm}{Wikipedia page}.
