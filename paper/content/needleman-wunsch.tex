\section{Needleman-Wunsch}
\label{sec:needleman-wunsch}
\newcommand{\lenn}{\text{len}}

The Needleman-Wunsch algorithm (developed by Saul B. Needleman and Christian D. Wunsch in 1970) calculates the global alignment of two strings and was originally used in bio-informatics to compare DNA sequences. For our purposes, the alphabet will instead consist of the phonetic \gls{ipa} symbols. Out of all possible alignments of two words (including gaps), the Needleman-Wunsch algorithm finds the one with the smallest distance, \ie the alignment with the highest \q{score}. The algorithm is based on dynamic programming and has a time complexity of $\mathcal{O}(\lenn(A) \cdot \lenn(B))$, where $A$ and $B$ are the two words to be compared.

\autoref{alg:needleman-wunsch} features the pseudo-code of the score computation\footnote{A good introduction to the algorithm can also be found on the respective \href{https://en.wikipedia.org/wiki/Needleman\%E2\%80\%93Wunsch_algorithm}{Wikipedia page}.}. In \autoref{fig:matrix-nuance-puissance-optimal}, we see the resulting score matrix that the algorithm constructed for the French words \textit{puissance} and \textit{nuance}. Follow the indicated path (red tiles) from the bottom right to the top left to find the (reversed) optimal alignment (see \autoref{tab:nuance-puissance-alignment-optimal});

\begin{table}[H]
\centering
\begin{tabular}{*{6}{>{\centering\arraybackslash}p{0.5cm}}}
    \toprule
    \textipa{n} & \textipa{\textturnh} & -- & -- & \textipa{\~A} & \textipa{s}\\
    \midrule
    \textipa{p} & \textipa{\textturnh} & \textipa{i} & \textipa{s} & \textipa{\~A} & \textipa{s}\\
    \bottomrule
\end{tabular}
\caption{The optimal alignment yields a score of $-2$. See the path in \autoref{fig:matrix-nuance-puissance-optimal}.}
\label{tab:nuance-puissance-alignment-optimal}
\end{table}


\begin{figure}[H]
    \centering
    \includegraphics[width=0.77\linewidth]{assets/illustrator/matrix-nuance-puissance-optimal.pdf}
    \caption{Needleman-Wunsch score matrix for the words $A\coloneqq\textit{puissance}$ \textipa{/p\textturnh is\~As/} (power, strength) and $B\coloneqq\textit{nuance}$ \textipa{/n\textturnh\~As/} (nuance, shade). The arrows indicate which steps locally maximize the score. Match Score: $1$, Mismatch Score: $-1$, Gap Penalty: $p=-2$.}
    \label{fig:matrix-nuance-puissance-optimal}
\end{figure}

\vfill\null

We first discuss the meaning of the different steps (arrows) in the score matrix (\autoref{fig:matrix-nuance-puissance-optimal}) to then explain how to construct this matrix.

\begin{itemize}

    \item In a \textbf{diagonal step}, both symbols that indicate the current position in the two words change. Such a step corresponds to either a match or a mismatch between the two symbols. In the example, the \textipa{/s/} symbols in the bottom-right corner match, which is why the step beforehand is a \textit{diagonal} step from the field $-3$ to $-2$. The score increases by $1$ since we defined the match score to be $+1$ (and a mismatch score as $-1$).

    \item In a \textbf{vertical} or \textbf{horizontal step}, only one of the two symbols changes. We interpret this as a gap in the alignment, \ie one symbol aligns to a gap in the other word. In the example, this is the case two times when we move from the red field $0$ down to $-2$ and then down to $-4$. The score decreases by $2$ each time, as we defined the gap penalty as $p \coloneqq -2$ in this example. The gap is indicated by \q{--} in the alignment (see \autoref{tab:nuance-puissance-alignment-optimal}). As we are still in the column of \textipa{/\textturnh/} of the word \textipa{/n\textturnh\~As/}, we insert two \q{--} symbols after the \textipa{/\textturnh/} in \autoref{tab:nuance-puissance-alignment-optimal}. This step is sometimes also referred to as \textbf{deletion} or \textbf{insertion}.
    
\end{itemize}

To find the score matrix for given input words $A$ and $B$, we follow \autoref{alg:needleman-wunsch}. First, the score matrix of dimension $(\lenn(A)+1) \times (\lenn(B)+1)$ is initialized\footnote{This does not necessarily involve setting all fields to $0$ as will become clear.}. Then, in lines~\ref{algstep:init-gap-start} to~\ref{algstep:init-gap-end}, the blue-bordered tiles of \autoref{fig:matrix-nuance-puissance-optimal} are filled with the gap penalty $p$ times the index. This is necessary since the only possible step for these tiles is either a vertical or horizontal step (blue arrows), thus leading to a gap in the alignment as discussed beforehand that we punish with the gap penalty $p$.

\begin{figure}[H]
    \centering
    \includegraphics[width=0.5\linewidth]{assets/illustrator/matrix-nuance-puissance-subcalc.pdf}
    \caption{Calculations for one element of the Needleman-Wunsch score matrix.}
    \label{fig:matrix-nuance-puissance-subcalc}
\end{figure}

In the nested loops (lines~\ref{algstep:nested1} and~\ref{algstep:nested2}), we then iterate over the remaining fields of the score matrix (index now starts at $1$, not $0$) which corresponds to traversing the matrix row-wise. To each field, we assign the maximum of three values:

\vfill\null
\columnbreak

\begin{algorithm*}
    \DontPrintSemicolon
    
    \SetKwFunction{calcScoreFunc}{calculateScore}
    \SetKwData{WordA}{$A$}
    \SetKwData{WordB}{$B$}
    \SetKwData{Sim}{similarity}
    \SetKwData{GapPenalty}{$p$}
    \SetKwData{ScoreMatrix}{scoreMatrix}
    \newcommand{\ScoreMatrixIdx}[2]{{\ScoreMatrix}[{#1}][{#2}]}
    \SetKwData{Cost}{cost}
    \SetKwData{MatchScore}{matchScore}
    \SetKwData{DeleteScore}{deleteScore}
    \SetKwData{InsertScore}{insertScore}
    \SetKwData{Score}{score}
    \SetKwFunction{len}{len}

    \KwIn{
        $\WordA = \set{A_0, \ldots, A_{\len(A)-1}}$,
        $\WordB = \set{B_0, \ldots, B_{\len(B)-1}}$,\\
        \qquad\qquad \Sim: similarityScoreFunc,
        \GapPenalty: GapPenalty
    }
    \KwOut{\Score}
    \Fn{\calcScoreFunc{}}
    {
        Init \ScoreMatrix with dimensions $(\len(\WordA)+1) \times (\len(\WordB)+1)$\;

        \BlankLine

        \For{$i \in \{0, \ldots, \len(\WordA)\}$\label{algstep:init-gap-start}}
        {
            \ScoreMatrixIdx{$i$}{$0$} $\gets \GapPenalty \cdot i$\;
        }
        \For{$j \in \{0, \ldots, \len(\WordB)\}$}
        {
            \ScoreMatrixIdx{$0$}{$j$} $\gets \GapPenalty \cdot j$
            \label{algstep:init-gap-end}\;
        }

        \BlankLine

        \For{$i \in \{1, \ldots, \len(\WordA)\}$\label{algstep:nested1}}
        {
            \For{$j \in \{1, \ldots, \len(\WordB)\}$\label{algstep:nested2}}
            {
                \Cost $\gets$ $\Sim(\WordA_{i-1}, \WordB_{j-1})$\;
                \MatchScore $\gets$ \ScoreMatrixIdx{$i-1$}{$j-1$} + \Cost
                \label{algstep:matchscore}\;
                \DeleteScore $\gets$ \ScoreMatrixIdx{$i-1$}{$j$} + \GapPenalty
                \label{algstep:deletescore}\;
                \InsertScore $\gets$ \ScoreMatrixIdx{$i$}{$j-1$} + \GapPenalty
                \label{algstep:insertscore}\;
                \ScoreMatrixIdx{$i$}{$j$} $\gets$ $\max(\MatchScore, \DeleteScore, \InsertScore)$\;
            }
        }

        \BlankLine

        \Return\! \ScoreMatrixIdx{$\len(\WordA)$}{$\len(\WordB)$}\;
    }
    
    \caption{Needleman-Wunsch}
    \label{alg:needleman-wunsch}
\end{algorithm*}

\begin{itemize}

    \item The \textbf{match score} is calculated by checking the step to the upper left diagonal (\autoref{algstep:matchscore}). In the example of \autoref{fig:matrix-nuance-puissance-subcalc}, this would result in a value $0+(-1) = -1$, where $0$ is the value in the upper left diagonal field and $-1$ is the cost of the mismatch between \textipa{/p/} and \textipa{/n/}. In case of a match, the new value would be $0 + 1 = 1$. In the algorithm, we also consider the case where costs for a match and a mismatch depend on the symbols themselves, which is why we introduce the function \tcboxverb{similarity} that returns the cost of aligning two symbols. This is especially useful when comparing phonetic symbols, as the similarity between two symbols can be defined in a more sophisticated way than just $1$ or $-1$ (\eg replacing a vowel with a consonant might be more costly than replacing a vowel with another vowel).
    
    \item The \textbf{delete score} refers to the step from the field above (\autoref{algstep:deletescore}). In the example, we find $(-2) + (-2) = -4$ as new value ($-2$ is the value in the field above and $p=-2$ is the gap penalty). This step corresponds to a gap in the alignment of word~$A$.
    
    \item The \textbf{insert score} refers to the step from the left (\autoref{algstep:insertscore}). In the example, we find $(-2) + (-2) = -4$ as new value ($-2$ is the value in the field to the left and $p=-2$ is the gap penalty). This step corresponds to a gap in the alignment of word~$B$.

\end{itemize}

The new value of the current field is then the maximum of the three values, such that we locally maximize the score: $\max(-1, -4, -4) = -1$. In \autoref{fig:matrix-nuance-puissance-optimal}, we additionally kept track of the steps that led to the optimal alignment by means of the rose arrows (here only the diagonal step yielding to a new score of $-1$).
