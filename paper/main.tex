\documentclass[10pt,a4paper,english]{article}
% adjust the page sizes to your likings
\usepackage[left=2.7cm,right=2.7cm,top=2cm,bottom=2cm]{geometry}
\usepackage[utf8]{inputenc}
\usepackage[USenglish]{babel}
\usepackage[T1]{fontenc} % allow west-european symbols (also umlauts: ä, ö, ü)
\usepackage{csquotes} % for \enquote{}
\usepackage{float} % for floating images
\usepackage[bottom]{footmisc} % keep footnotes at the bottom of the page
\usepackage{setspace} % to have the \setstretch{baselinestrech} command available
\usepackage{ragged2e} % text alignment
\usepackage[parfill]{parskip} % no indentation for paragraphs, instead add spacing
\usepackage{pdfpages} %for including pdf

\usepackage{tipa} % for phonetic symbols

% Easy tables, e.g. for coloring columns as seen here.
% https://tex.stackexchange.com/a/613906/
% \usepackage{tabularray}
\usepackage{subfigure}
\usepackage{booktabs}

\usepackage{abstract}
\usepackage{multicol}

% Graphics & Color
\usepackage{xcolor}
\definecolor{heidelberg-red}{HTML}{E50A37}
\usepackage{graphicx}
\usepackage{caption}

% Math & Physics
\usepackage{amsmath}
\usepackage{amsfonts}
\usepackage{amsthm}
\usepackage{amssymb}
\usepackage{bm}
\usepackage{cancel}

\usepackage{witharrows} % http://mirrors.ibiblio.org/CTAN/macros/generic/witharrows/witharrows.pdf
\usepackage{nicefrac} % usage: \nicefrac{nominator}{denominator}
\usepackage{physics}
\usepackage{siunitx}
\sisetup{
	locale=US,
	group-separator={,},
	group-digits=integer,
	quotient-mode=fraction,per-mode=symbol}
\usepackage{empheq} % boxed equations: https://trucastuces.wordpress.com/2012/10/10/boxed-equations-in-latex/

% Acronyms
\usepackage[automake]{glossaries}
\setacronymstyle{long-short}

\usepackage{lmodern}
\usepackage{fancyhdr}
\usepackage{enumerate}
% we use the option hidelinks here to avoid ugly borders around clickable items, e.g. links and cross-references
\usepackage{hyperref} % clickable links
\hypersetup{
	colorlinks = true,
	linkcolor={red!50!black}, % Color of internal links
	citecolor={blue!50!black}, % Color of citations
	urlcolor={blue!80!black} % Color for external hyperlinks
}
\usepackage{todonotes}
\setlength {\marginparwidth }{2cm}      % margin for todo notes
\usepackage[nameinlink]{cleveref}

\let\qty\SI % see "bug" (warning) in https://tex.stackexchange.com/a/628184

% Enumerations
\usepackage{enumitem}
% Enumerate lists as a) b) instead of 1. 2.
\setenumerate[0]{label=\textbf{\alph*)}}
% remove indentation of enumerate environment
\setlist{leftmargin=*}

% ---- Algorithms ----
% using algorithm2e, the third algorithm typesetting environment
%\usepackage{algorithm}                  % Für Algorithmen-Umgebung (ähnlich wie lstlistings Umgebung)
%\usepackage{algpseudocode}              % Für Pseudocode. Füge "[noend]" hinzu, wenn du kein "endif",
% etc. haben willst.
\usepackage{float}
\usepackage{setspace}
% http://tug.ctan.org/macros/latex/contrib/algorithm2e/doc/algorithm2e.pdf
\usepackage[ruled, algosection, lined, scleft, linesnumbered, noend]{algorithm2e}
\renewcommand{\algorithmautorefname}{Algorithm}

% Algorithm comments
\newcommand\mycommentsfont[1]{\footnotesize\ttfamily{#1}}
\SetCommentSty{mycommentsfont} % https://tex.stackexchange.com/a/162208/249769
\SetKwComment{Comment}{$\triangleright$\ }{} % https://tex.stackexchange.com/a/200736/249769
\SetKwRepeat{DoWhile}{do}{while}

% Function/Procedure
\SetKwProg{Fn}{Function}{:}{}
\SetArgSty{textnormal}

% Keywords
\SetKw{Continue}{continue} % https://tex.stackexchange.com/a/298503/249769
\SetKw{Break}{break}

% Code
% https://stackoverflow.com/a/79057262/
\usepackage[most]{tcolorbox}
\tcbset{
    frame empty,
    colback=lightgray!20
}

% Bibliography
\usepackage[
    backend=biber,
    style=numeric
]{biblatex}

\addbibresource{literature.bib}

% Abbreviations
\newcommand{\eg}{e.\,g. }
\newcommand{\ie}{i.\,e. }

% \q{I am citing} in order to get "I am citing"
\newcommand{\q}[1]{\enquote{#1}}

% Math
\newcommand{\set}[1]{\ensuremath{\{\,#1\,\}}}
\renewcommand{\implies}{\Rightarrow}
\renewcommand{\iff}{\Leftrightarrow}

% Coding
\newcommand\Cpp{C\nolinebreak[4]\hspace{-.05em}\raisebox{.4ex}{\relsize{-3}{\textbf{++}}}}

\input{prettify-math}

\renewcommand{\abstractnamefont}{\normalfont\bfseries}
\renewcommand{\abstracttextfont}{\normalfont\small\itshape}

\title{\vspace{-0.0em}Parallel Needleman-Wunsch on CUDA to measure word similarity based on phonetic transcriptions}
\author{Dominic Plein}
\date{February 19, 2025}

\newcommand{\abstractText}{\noindent
	\newline\noindent
    We present a method to calculate the similarity between words based on their phonetic transcription using the Needleman-Wunsch algorithm. We implement this algorithm in Rust and parallelize it on both CPU and GPU to handle large datasets efficiently. The GPU implementation leverages CUDA and the cudarc Rust library to achieve significant performance improvements. We validate our approach by constructing a fully-connected graph where nodes represent words and edge have weights according to the similarity between the words. This graph is then analyzed using clustering algorithms to identify groups of phonetically similar words. Our results demonstrate the feasibility and effectiveness of the proposed method in analyzing the phonetic structure of languages. It might be easily expanded to other languages.
}

% Acronyms
\makeglossaries
\newacronym{ipa}{IPA}{International Phonetic Alphabet}
\newacronym{pos}{POS}{Part-of-Speech}
\newacronym{ptx}{PTX}{Parallel Thread Execution}

\begin{document}

\setlength{\abovedisplayskip}{0.2em}
%\setlength{\belowdisplayskip}{0pt}
%\setlength{\abovedisplayshortskip}{0pt}
%\setlength{\belowdisplayshortskip}{0pt}

% Title & Abstract
\maketitle
\begin{center}
    \small{Part of the GP-GPU Computing course at UGA by Christophe Picard}
\end{center}

\vspace{1em}
\begin{abstract}
    \abstractText
    \newline
    \newline
\end{abstract}
% \twocolumn[
%   \begin{@twocolumnfalse}
%     \maketitle
%     \begin{abstract}
%       \abstractText
%       \newline
%       \newline
%     \end{abstract}
%   \end{@twocolumnfalse}
% ]

% Content
% \setcounter{section}{1}

% \pagebreak

\begin{multicols*}{2}
\tableofcontents
\section{Introduction}

The \gls{ipa} uses special symbols\footnote{See for example the French list \href{https://en.wikipedia.org/wiki/Help:IPA/French}{here}.} to represent the sound of a spoken language \cite{ipa}. This is useful for language learners since the pronunciation of a word can be significantly different from its written form. For example, the French word \textit{renseignement} (information) is pronounced \textipa{/K\~{a}.sE\textltailn.m\~{a}/}. Based on this alphabet, one might wonder if we can construct a metric that quantifies the \textbf{similarity between two words based on their phonetic transcription}. This would allow to construct a graph where nodes are words and undirected edges are weighted by the distance between the words. Such a graph can be used to find neighbors of a word based on the respective phonetic similarity. This opens up the possibility to apply clustering algorithms and other methods stemming from graph theory in order to analyze the phonetic structure of a language.

Calculating the distance between each pair of words corresponds to a fully-connected graph. Our dataset consists of around 600,000 French words and their \gls{ipa} transcription, alongside their frequency in the French language. Excluding self-loops, we find a vast number of edges \eqref{eq:num-edges}:
\begin{align}
    \text{\#nodes} &= 600,000 \\
    \text{\#edges} &= \frac{600,000 \cdot 599,999}{2} \approx \num{1.8e11}
\end{align}
This high number and the independent nature of the distance calculation for each pair of words makes the problem well-suited for parallelization. In \autoref{sec:needleman-wunsch}, we present the Needleman–Wunsch algorithm used to calculate the distance between two words. In \autoref{sec:impl}, we discuss how to parallelize this algorithm on a CPU using the \textit{Rayon} library in Rust and on a consumer Nvidia GPU using the CUDA framework with the \textit{cudarc} Rust library. Finally, we present performance results and provide visualizations of the obtained graphs in \autoref{sec:eval}. We conclude in \autoref{sec:conclusion} and discuss how this method can be improved and how future work could extend it.

% In the following, by \textit{word} we always refer to its phonetic transcription. That is, homophones like the French words \textit{vert} \textipa{/vEK/} (green) and \textit{verre} \textipa{/vEK/} (glass) are considered the same word.

% \vfill\null

\newcolumn
\section{Needleman-Wunsch}
\label{sec:needleman-wunsch}
\newcommand{\lenn}{\text{len}}

The Needleman-Wunsch algorithm (developed by Saul B. Needleman and Christian D. Wunsch in 1970) calculates the global alignment of two strings and was originally used in bio-informatics to compare DNA sequences. For our purposes, the alphabet will instead consist of the phonetic \gls{ipa} symbols. Out of all possible alignments of two words (including gaps), the Needleman-Wunsch algorithm finds the one with the smallest distance, \ie the alignment with the highest \q{score}. The algorithm is based on dynamic programming and has a time complexity of $\mathcal{O}(\lenn(A) \cdot \lenn(B))$, where $A$ and $B$ are the two words to be compared.

\autoref{alg:needleman-wunsch} features the pseudo-code of the score computation\footnote{The respective \href{https://en.wikipedia.org/wiki/Needleman\%E2\%80\%93Wunsch_algorithm}{Wikipedia page} also provides a good introduction. Furthermore, the score matrix is interactively explained in the \href{https://bioboot.github.io/bimm143_W20/class-material/nw/}{Global Alignment App}.}. In \autoref{fig:matrix-nuance-puissance-optimal}, we see the resulting score matrix that the algorithm constructed for the French words \textit{puissance} and \textit{nuance}. Follow the indicated path (red tiles) from the bottom right to the top left to find the (reversed) optimal alignment (see \autoref{tab:nuance-puissance-alignment-optimal});

\begin{table}[H]
\centering
\begin{tabular}{l*{6}{>{\centering\arraybackslash}p{0.2cm}}}
    \toprule
    \textit{puissance}
    & \textipa{p} & \textipa{\textturnh} & \textipa{i} & \textipa{s} & \textipa{\~A} & \textipa{s}\\
    \midrule
    \textit{nuance}
    & \textipa{n} & \textipa{\textturnh} & -- & -- & \textipa{\~A} & \textipa{s}\\
    \bottomrule
\end{tabular}
\caption{The optimal alignment yields a score of $-2$. See the path in \autoref{fig:matrix-nuance-puissance-optimal}.}
\label{tab:nuance-puissance-alignment-optimal}
\end{table}

\begin{figure}[H]
    \centering
    \includegraphics[width=0.77\linewidth]{assets/illustrator/matrix-nuance-puissance-optimal.pdf}
    \caption{Needleman-Wunsch score matrix for the words $A\coloneqq\textit{puissance}$ \textipa{/p\textturnh is\~As/} (power, strength) and $B\coloneqq\textit{nuance}$ \textipa{/n\textturnh\~As/} (nuance, shade). The arrows indicate which steps locally maximize the score. The red tiles trace the path of the optimal alignment. Match Score: $1$, Mismatch Score: $-1$, Gap Penalty: $p=-2$.}
    \label{fig:matrix-nuance-puissance-optimal}
\end{figure}

\vfill\null

We first discuss the meaning of the different steps (arrows) in the score matrix (\autoref{fig:matrix-nuance-puissance-optimal}) to then explain how to construct this matrix.

\begin{itemize}

    \item In a \textbf{diagonal step}, both symbols that indicate the current position in the two words change. Such a step corresponds to either a match or a mismatch between the two symbols. In the example, the \textipa{/s/} symbols in the bottom-right corner match, which is why the step beforehand is a \textit{diagonal} step from the field $-3$ to $-2$. The score increases by $1$ since we defined the match score to be $+1$ (and a mismatch score as $-1$).

    \item In a \textbf{vertical} or \textbf{horizontal step}, only one of the two symbols changes. We interpret this as a gap in the alignment, \ie one symbol aligns to a gap in the other word. In the example, this is the case two times when we move from the red field $0$ down to $-2$ and then down to $-4$. The score decreases by $2$ each time, as we defined the gap penalty as $p \coloneqq -2$ in this example. The gap is indicated by \q{--} in the alignment (see \autoref{tab:nuance-puissance-alignment-optimal}). As we are still in the column of \textipa{/\textturnh/} of the word \textipa{/n\textturnh\~As/}, we insert two \q{--} symbols after the \textipa{/\textturnh/} in \autoref{tab:nuance-puissance-alignment-optimal}. This step is sometimes also referred to as \textbf{deletion} or \textbf{insertion}.
    
\end{itemize}

To find the score matrix for given input words $A$ and $B$, we follow \autoref{alg:needleman-wunsch}. First, the score matrix of dimension $(\lenn(A)+1) \times (\lenn(B)+1)$ is initialized\footnote{This does not necessarily involve setting all fields to $0$ as will become clear.}. Then, in lines~\ref{algstep:init-gap-start} to~\ref{algstep:init-gap-end}, the blue-bordered tiles of \autoref{fig:matrix-nuance-puissance-optimal} are filled with the gap penalty $p$ times the index. This is necessary since the only possible step for these tiles is either a vertical or horizontal step (blue arrows), thus leading to a gap in the alignment as discussed beforehand that we punish with the gap penalty $p$.

\begin{figure}[H]
    \centering
    \includegraphics[width=0.55\linewidth]{assets/illustrator/matrix-nuance-puissance-subcalc.pdf}
    \caption{Calculations for one element of the Needleman-Wunsch score matrix.}
    \label{fig:matrix-nuance-puissance-subcalc}
\end{figure}

In the nested loops (lines~\ref{algstep:nested1} and~\ref{algstep:nested2}), we then iterate over the remaining fields of the score matrix (index now starts at $1$, not $0$) which corresponds to traversing the matrix row-wise. To each field, we assign the maximum of three values:

% \vfill\null
% \columnbreak

\begin{figure*}

\begin{minipage}[t]{0.6\textwidth}

\begin{algorithm}[H]
    \DontPrintSemicolon
    
    \SetKwFunction{calcScoreFunc}{calculateScore}
    \SetKwData{WordA}{$A$}
    \SetKwData{WordB}{$B$}
    \SetKwData{Sim}{similarity}
    \SetKwData{GapPenalty}{$p$}
    \SetKwData{ScoreMatrix}{scoreMatrix}
    \newcommand{\ScoreMatrixIdx}[2]{{\ScoreMatrix}[{#1}][{#2}]}
    \SetKwData{Cost}{cost}
    \SetKwData{MatchScore}{matchScore}
    \SetKwData{DeleteScore}{deleteScore}
    \SetKwData{InsertScore}{insertScore}
    \SetKwData{Score}{score}
    \SetKwFunction{len}{len}

    \KwIn{
        \small{\!\!\small{$\WordA = \set{A_0, \ldots, A_{\len(A)-1}}$,
        $\WordB = \set{B_0, \ldots, B_{\len(B)-1}}$}},\\
        \qquad\qquad \Sim: \small{similarityScoreFunc},
        \GapPenalty: GapPenalty
    }
    \KwOut{\Score}
    \Fn{\calcScoreFunc{}}
    {
        \SetInd{0.25em}{0.55em}
        Init \ScoreMatrix with dimensions $\small{(\len(\WordA)+1) \times (\len(\WordB)+1)}$\;

        \BlankLine

        \For{$i \in \{0, \ldots, \len(\WordA)\}$\label{algstep:init-gap-start}}
        {
            \ScoreMatrixIdx{$i$}{$0$} $\gets \GapPenalty \cdot i$\;
        }
        \For{$j \in \{0, \ldots, \len(\WordB)\}$}
        {
            \ScoreMatrixIdx{$0$}{$j$} $\gets \GapPenalty \cdot j$
            \label{algstep:init-gap-end}\;
        }

        \BlankLine

        \For{$i \in \{1, \ldots, \len(\WordA)\}$\label{algstep:nested1}}
        {
            \For{$j \in \{1, \ldots, \len(\WordB)\}$\label{algstep:nested2}}
            {
                \Cost $\gets$ $\Sim(\WordA_{i-1}, \WordB_{j-1})$
                \label{algstep:sim}\;
                \MatchScore $\gets$ \ScoreMatrixIdx{$i-1$}{$j-1$} + \Cost
                \label{algstep:matchscore}\;
                \DeleteScore $\gets$ \ScoreMatrixIdx{$i-1$}{$j$} + \GapPenalty
                \label{algstep:deletescore}\;
                \InsertScore $\gets$ \ScoreMatrixIdx{$i$}{$j-1$} + \GapPenalty
                \label{algstep:insertscore}\;
                \ScoreMatrixIdx{$i$}{$j$} $\gets$
                $\max(\MatchScore, \DeleteScore, \InsertScore)$
                \label{algstep:max}\;
            }
        }

        \BlankLine

        \Return\! \ScoreMatrixIdx{$\len(\WordA)$}{$\len(\WordB)$}\;
    }
    
    \caption{Needleman-Wunsch}
    \label{alg:needleman-wunsch}
\end{algorithm}

\end{minipage}%
\hfill
\begin{minipage}[t]{0.35\textwidth}

\begin{figure}[H]
    \centering
    \includegraphics[width=0.87\linewidth]{assets/illustrator/matrix-nuance-puissance-non-optimal.pdf}
    \caption{Needleman-Wunsch score matrix and the path (in red) for a non-optimal alignment. Orange strokes indicate non-optimal steps. Parameters as in \autoref{fig:matrix-nuance-puissance-optimal}.}
    \label{fig:matrix-nuance-puissance-non-optimal}
\end{figure}

\end{minipage}
\end{figure*}

% \columnbreak

\begin{itemize}[leftmargin=0cm]

    \item The \textbf{match score} is calculated by checking the step to the upper left diagonal (\autoref{algstep:matchscore}). In the example of \autoref{fig:matrix-nuance-puissance-subcalc}, this would result in a value $0+(-1) = -1$, where $0$ is the value in the upper left diagonal field and $-1$ is the cost of the mismatch between \textipa{/p/} and \textipa{/n/}. In case of a match, the new value would be $0 + 1 = 1$. In the algorithm, we also consider the case where costs for a match and a mismatch depend on the symbols themselves, which is why we introduce the function \tcboxverb{similarity} that returns the cost of aligning two symbols. This is especially useful when comparing phonetic symbols, as the similarity between two symbols can be defined in a more sophisticated way than just $1$ or $-1$ (\eg replacing a vowel with a consonant might be more costly than replacing a vowel with another vowel).
    
    \item The \textbf{delete score} refers to the step from the field above (\autoref{algstep:deletescore}). In the example, we find $(-2) + (-2) = -4$ as new value ($-2$ is the value in the field above and $p=-2$ is the gap penalty). This steps signifies that a symbol in word A aligns to a gap in word B (here: \textipa{/i/} and \textipa{/s/} of \textit{puissance} align to gaps in \textit{nuance}).
    
    \item The \textbf{insert score} refers to the step from the left (\autoref{algstep:insertscore}). In the example, we find $(-2) + (-2) = -4$ as new value ($-2$ is the value in the field to the left and $p=-2$ is the gap penalty). This steps signifies that a symbol in word B aligns to a gap in word A (this does not occur in the example).

\end{itemize}

The new value of the current field is assigned the maximum of these values (\autoref{algstep:max}), such that we locally maximize the score: $\max(-1, -4, -4) = -1$. In \autoref{fig:matrix-nuance-puissance-optimal}, we additionally kept track of the steps that led to the optimal alignment by means of the rose arrows (here only the diagonal step that yields the new maximal score of $-1$). For our purposes, we don't want to reconstruct the exact alignment that led to the optimal score, but only the score itself. Thus, we can omit the backtracking step and don't need to store the rose arrows.

By construction, \textbf{the bottom-right field of the score matrix contains the score of the optimal alignment}. This is ensured by the Principle of Optimality (Bellman), which states that an optimal solution to a problem can be constructed from optimal solutions to its subproblems. In the context of the Needleman-Wunsch algorithm, this means that the optimal alignment score for two sequences (words) can be derived by considering the optimal alignment scores of progressively smaller subsequences. Each cell in the score matrix represents the optimal score for the corresponding prefixes of the two words up to that point, since we take the maximum of the three possible steps (match, delete, insert) at each cell. This ensures that the final cell (in the bottom right) contains the optimal score for the entire sequences.

\autoref{tab:nuance-puissance-alignment-non-optimal} shows an example of a non-optimal alignment of the two words, yielding a score of $-15$ (compared to $-2$ for the optimal path). \autoref{fig:matrix-nuance-puissance-non-optimal} depicts the corresponding score matrix. Note how the indicated path includes 4 non-optimal choices (orange strokes).

% \vspace{-0.65em}

\begin{table}[H]
    \centering
    \begin{tabular}{l*{9}{>{\centering\arraybackslash}p{0.2cm}}}
        \toprule
        \textit{puissance}
        & -- & \textipa{p} & \textipa{\textturnh} & -- & \textipa{i}
        & -- & \textipa{s} & \textipa{\~A} & \textipa{s}\\
        \midrule
        \textit{nuance}
        & \textipa{n} & -- & \textipa{\textturnh} & \textipa{\~A} & --
        & \textipa{s} & -- & -- & --\\
        \bottomrule
    \end{tabular}
    \caption{This non-optimal alignment yields a score of $-15$. See the path in \autoref{fig:matrix-nuance-puissance-non-optimal}.}
    \label{tab:nuance-puissance-alignment-non-optimal}
\end{table}

\begin{figure*}
    \centering
    \includegraphics[width=0.8\textwidth]{assets/illustrator/traverse-schema.pdf}
    \caption{Schema of the traversal of the adjacency matrix.}
    \label{fig:traverse-schema}
\end{figure*}

\section{Data Preparation}
\label{sec:data}

We use the \href{https://github.com/frodonh/french-words}{\textbf{french-words}} dataset (frodonh, 2020), which contains 691,969 French words. It is compiled from several sources including (among others) the Debian package \href{https://packages.debian.org/fr/sid/wfrench}{wfrench} (used for spell checking), \href{http://www.lexique.org/}{Lexique 3.83} (Boris New \& Christophe Pallier), the \href{https://infolingu.univ-mlv.fr/DonneesLinguistiques/Dictionnaires/telechargement.html}{DELA dictionary} from the University of Marne-la-Vallée as well as the \href{https://github.com/hbenbel/French-Dictionary}{French-Dictionary} (Hussem Ben Belgacem). The words also include \acrfull{pos} tagging information, \eg whether it is a noun, verb, adjective, preposition etc. It also comprises the usage frequency according to Lexique.org and Google Ngrams. We use the average of both sources (or just one if the other is missing).

Since the french-words dataset does not include the phonetic transcriptions, we consult the \href{https://github.com/DanielSWolf/wiki-pronunciation-dict}{wiki-pronunciation-dict} (Daniel Wolf, 2021) extracted from the French \href{https://fr.wiktionary.org/}{Wiktionnaire}. We merge both datasets to obtain 611,786 words with their \gls{ipa} transcription. If multiple phonetic transcription are available, we only store the first one. For easy access, we store the data in a dataclass and serialize it to a pickle file of around \qty{60}{\mega\byte}.

Additionally, we extract all used phonetic symbols and assign integer IDs to them. This enables us to store the transcription as a list of integers. For our examples, we obtain \autoref{tab:phonetic-encoding}. The Needleman-Wunsch algorithm will then work on these integer lists. Note that we consider \textipa{/dZ/} as one symbol, even though it is a combination of \textipa{/d/} and \textipa{/Z/}. The same applies to \textipa{/tS/}. This is to account for the different pronunciation of the combined symbols compared to the individual ones.

\vspace{-0.4em}

\begin{table}[H]
    \centering
    \begin{tabular}{lll}
    \toprule
    \textbf{Word} & \textbf{\acrshort{ipa}} & \textbf{Encoding} \\
    \midrule
    \textit{puissance} & \textipa{/p\textturnh is\~As/} & $[0,18,16,11,26,11]$ \\
    \textit{nuance} & \textipa{/n\textturnh\~As/} & $[29,18,26,11]$ \\
    \bottomrule
    \end{tabular}
    \caption{Example of two words with their phonetic transcription and encoding.}
    \label{tab:phonetic-encoding}
\end{table}

\vfill\null
\columnbreak
\section{Parallelized algorithms}
\label{sec:impl}

TODO

\section{Evaluation}
\label{sec:eval}

We deploy our GPU code on a consumer Nvidia GeForce GTX 1060\footnote{We use the Driver Version 572.42 and CUDA Toolkit 12.8 inside WSL2 (Ubuntu 22.04 jammy).} with 6GB GDDR5. Every code change related to the GPU code is verified by comparing the resulting binary edge weights file with the one generated by our parallelized Rust implementation on the CPU. This baseline helps to quickly identify errors, which could otherwise remain unnoticed. During our tests, we define a manual threshold to cap the number of words to a user-defined threshold. The words considered are sorted according to their frequency as we are interested in relationships between the most commonly used words.

\textbf{Performance.} A fair comparison between the CPU and GPU implementation is not possible since focus was put in optimizing the GPU code. For example, the CPU implementation currently stores the row and column number alongside the actual edge weight in RAM to be able to order the results to the row-major ordering afterwards. To give an order of magnitude, the parallelized Rust CPU implementation (without the subsequent sorting) takes around $\qty{12}{\s}$ (for 10,000 nodes), \qty{42}{\s} (for 20,000 nodes) and \qty{93}{\s} for 30,000 nodes on a 4-core Intel i7-6700 CPU. The implementation is limited to around 35,000 words when $\approx\qty{20}{\giga\byte}$ of RAM are available.

\begin{figure}[H]
    \centering
    \includegraphics[width=\linewidth]{assets/timing.pdf}
    \caption{Performance of the GPU code for different number of nodes $n$. Number of edges via \eqref{eq:num-edges}.}
    \label{fig:timing}
\end{figure}

\vspace{-1.5em}

To test the performance of the GPU code, we measure the kernel execution time (including copying the results back to the host) for a range of number of nodes $n$ in the graph. For every $n$, we measure the duration 12 times\footnote{After every run, the device is re-initialized. Furthermore, we wait \qty{2}{\s} after every run before a new one starts.} and calculate mean and variance. The results are depicted in \autoref{fig:timing}. The variance is not shown as it is too small to be visible (always less than \qty{1}{\ms}). For $20,000$ words, the GPU code takes $\qty{484}{\ms}$ on average, while the CPU implementation needs $\qty{42}{\s}$. Up to $100,000$ nodes (\ie up to almost 5 billion edges), the GPU implementation takes less than \qty{20}{\s}. \autoref{fig:timing} also reveals the linear trend of time with increasing number of edges, which was to be expected since a thread is launched for every edge.

Our CUDA implementation is limited by the global memory (\qty{6}{\giga\byte} for the GPU at hand). This memory is used to store the resulting edge weights, \ie one byte per edge. The maximum number of edges we can handle is therefore the available memory divided by 1~byte. To obtain the corresponding number of nodes, we solve \eqref{eq:num-edges} for~$n$:
\begin{align}
    n = \frac{1}{2} + \sqrt{\frac{1}{4} + 2 \cdot \text{num edges}}
\end{align}
On the GPU at hand, we can handle up to around 107,000 nodes (mean time $\approx \qty{21.3}{\s}$) before experiencing \q{CUDA out of memory errors}. Currently we detect the limit, but do not implement a mechanism to go beyond it. One way could to be to detect the error, then copy the results back to the host and continue the computation while shifting the index back to $0$. The results are then concatenated on the host. For the further evaluation, we shall content ourselves with the results for the first 100,000 words, which already contain a lot of information of the French language. The binary file holding only the edge weights in row-major order, is \qty{4.66}{\giga\byte} in size.

\textbf{Graph application.} In a further post-processing step, we convert the binary edge file to a CSV file and import it into the open-source graph visualization software \href{https://gephi.org/}{Gephi}. During the generation of the CSV file, we filter out edges with a weight below a certain threshold since Gephi is not able to handle the amount of data otherwise. \autoref{fig:edge-weight-histogram} depicts the histogram of edge weights.

\vfill
\null

\begin{figure}[H]
    \centering
    \includegraphics[width=\linewidth]{assets/edge_weights.pdf}
    \caption{Histogram of edge weights (linear and logarithmic scale). Mean: $-1.5$, range: $[-19, 16]$.}
    \label{fig:edge-weight-histogram}
\end{figure}



% Based on the first 100,000 most frequently used words, we calculated the distance between every pair and visualized the graph using \href{https://gephi.org/}{Gephi's} ForceAtlas2 algorithm. The neighbors of the word \q{glace} and \q{prévoir} are shown in \autoref{fig:neighbors}. For bigger graphs than that, Gephi is not able to handle the amount of data anymore.

% Having translated the problem into a graph structure also allows us to use graph algorithms to discover interesting properties. As an example, Gephi implements the \textit{shortest path algorithm}: users can click on two words and the shortest path between them is calculated and shown in the graph. Beforehand, we filtered the graph to only include the most strong edges. With this, we can find chains like the following (read them aloud to hear the phonetic similarity):
% \begin{itemize}
%     \item trottoir $\rightarrow$ entrevoir $\rightarrow$ devoir $\rightarrow$ voire $\rightarrow$ voile $\rightarrow$ val $\rightarrow$ valait $\rightarrow$ fallait $\rightarrow$ falaise
%     \item falaise $\rightarrow$ fallait $\rightarrow$ palais $\rightarrow$ passais $\rightarrow$ dépassait $\rightarrow$ dépendait $\rightarrow$ répondait $\rightarrow$ répond $\rightarrow$ raison $\rightarrow$ maison
%     \item confusion $\rightarrow$ conclusion $\rightarrow$ exclusion $\rightarrow$ explosion $\rightarrow$ exposition $\rightarrow$ explications $\rightarrow$ respiration $\rightarrow$ précipitation $\rightarrow$ présentation $\rightarrow$ présenta $\rightarrow$ présente $\rightarrow$ présence $\rightarrow$ présidence $\rightarrow$ résidence $\rightarrow$ résistance $\rightarrow$ existence
% \end{itemize}

\section{Conclusion \& Outlook}
\label{sec:conclusion}

We presented a method calculate the similarity between words based on their phonetic \gls{ipa} transcription. For this purpose, we employed the Needleman-Wunsch algorithm with similarity matrix and gap penalty. We implemented this algorithm in Rust and parallelized it on a CPU using the Rayon library and on a consumer Nvidia GPU using the CUDA framework with the \textit{cudarc} Rust library, while writing the kernel itself in \Cpp. We detailed choice of data structures and memory layout to optimize the performance of the GPU implementation. For a graph with 100,000 nodes (words) and almost 5 billion edges (word-pairs), the algorithm takes less than \qty{20}{\s}. consistency between the CPU and GPU implementations was verified up to 30,000 nodes (after that the CPU implementation becomes too slow and consumes too much memory).

Future work could include an adapted version that can read the whole graph of more than 600,000 nodes at the same time by copying back intermediate results from the GPU to the CPU as already outlined in the evaluation. To further speed up the computation, one could consider a more fine-grained parallelization, \ie parallelize at the level of the score matrix calculation. This is not trivial due to the dependencies between the cells of the matrix. A stencil computation approach could be used to parallelize this part of the algorithm.

We also demonstrated the practical usability of the resulting edge weights by examining the graphs in Gephi. Even just by looking at small subsections of the full graph, we can deduce interesting properties of the language, \eg community detection revealed groups with similar endings and groups with the same root but different endings. Constructing the ego-network of a word allows to find words that are phonetically similar to a given word. One can envision an online platform where this functionality is offered to users to find rhymes or similar sounding words. This can be helpful for language learners as well to foster the playful exploration of a language. Future work could include:

\begin{itemize}
    \item Fine-tune the similarity matrix for language subtleties. In this document, we always used a fixed match/mismatch score of 1/-1, while in reality, some phonetic substitutions sound more similar than others. For example, replacing a vowel by a consonant might be more severe than replacing a vowel by another vowel. One might also want to experiment with the gap penalty that was set to $p=-1$ throughout.
    
    \item So far, we considered a dataset for the French language. The presented pipeline can work with other languages as well. This can open up the possibility to compare the phonetic structure of different languages and find similarities between them.
    
    \item We already showcased how a shortest path can find chains of words that sound similar and still allow to go from one word to another. Other graph-theoretical methods could be applied as well. For example, the Louvain method inherently constructs a multi-level hierarchy of communities. Looking at a lower granularity level (more high-level view) could reveal interesting results. For this purpose, we should consider a parallel Louvain implementation that can work on the whole graph. Limiting oneself to a slice of edge weights (as done here) might bias results or leave interesting structures unnoticed.
    
    % todo roots, lexeme
\end{itemize}


% Domain Level
% - offer online service to show neighbors of a word in a ego network
%   (users can search for a word)
% - more graph theory applications other than shortest path?
% normalize score to word length n+m
% todo lemma, lexeme
% https://en.wikipedia.org/wiki/Lemma_%28morphology%29

% Computation
% - stencil computation -> try to parallelize at finer granularity
% - Calculate more words at the same time by filtering uninteresting weights
%   directly (for this purpose generate histogram of weights distribution)

\end{multicols*}

\printglossary[type=\acronymtype]

\printbibliography[
    heading=bibintoc,
    title={Bibliography},
    keyword={lit}
]
\printbibliography[
    title={Data sources},
    keyword={data}
]

\end{document}
